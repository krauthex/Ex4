\documentclass[Ex4_Zusammenfassung.tex]{subfiles}


\begin{document}

\section{Darstellung der Teilchen-WW mit Feynman-Diagrammen}
\textbf{von \soeren \& \martina}\\

Zur Berechnung der Wirkungsquerschnitte bzw. Übergangswahrscheinlichkeiten gibt es in der Quantenelektrodynamik (QED) Potenzreihen, die sich durch sogenannte Feynman--Diagramme verbildlichen lassen. Wir werden uns damit beschäftigen, welche Wechselwirkungen überhaupt möglich sind und die niedrigsten Ordnungen davon auch aufzeichnen. Wir kennen stets nur die Reaktionsedukte und --produkte und bauen uns daraus folgendes Modell auf:  

\subsection{Regeln}
	
	\begin{itemize}
		\item Meist ist die Zeitachse eingezeichnet, übrige Richtungen beziehen sich auf den Raum
		
		\item Teilchen werden durch Linien symbolisiert, dabei gibt die Pfeilrichtung an, ob es sich um das Teilchen selbst 	(Pfeil in Richtung der Zeitachse) oder sein Antiteilchen (Pfeil entgegengesetzt zur Zeitachse gerichtet).\\
		Die reellen Teilchen erfüllen alle $E^2 - p^2 = m^2\ (c=1)$. 
		
		\item virtuelle Teilchen hingegen sind durch im Feynman--Diagramm abgeschlossene Linie dargestellt und erfüllen $E^2 - p^2 = m^2$ \textbf{nicht}!
		
		\item Linienformen können angeben, um welche Teilchen es sich handelt. So haben Fermionen (Quarks, Leptonen) eine durchgezogene Linie, die Eichbosonen $\gamma,\ W^+,\ W^-,\ Z^0$ allgemein eine Wellenlinie und Gluonen eine schraubenförmige Linie. Oftmals werden aber auch geschtrichelte Linien für Propagatoren der schwachen Wechselwirkung und die Wellenlinien nur für Photonen verwendet (so wird es auch in diesem Skript gehandhabt).
		\begin{figure}[H]
			\centering
			\qquad 
			\begin{subfigure}[b]{0.3\textwidth}
				\feynmandiagram [horizontal=f1 to f2] {%
					f1 -- [photon] f2,
				};
				\subcaption{Photon}
			\end{subfigure}
			\begin{subfigure}[b]{0.3\textwidth}
				\feynmandiagram [horizontal=f1 to f2] {%
					f1 -- [scalar] f2,
				};
				\subcaption{Eichboson}
			\end{subfigure}
			\begin{subfigure}[b]{0.3\textwidth}
				\feynmandiagram [horizontal=f1 to f2] {%
					f1 -- [gluon] f2,
				};
				\subcaption{Gluon}
			\end{subfigure}
		\end{figure}
		
		\item Vertices, die Knotenpunkte zwischen den Linien, geben durch ihre Anzahl die Ordnung der Feynman--Diagramme an. Sie verbinden (in echten Feynman--Diagrammen) genau drei Linien. 
		
		\item Propagatoren, die Linien zwischen Vertices, sind virtuell und haben keine Vorzugsrichtung.
	\end{itemize}
	
\begin{table}[H]
	\centering
	$
	\begin{array}{c|ccc}
		\textbf{Eichboson} & \textbf{Anzahl} & \textbf{WW} & \textbf{auf Materieteilchen} \\
		\hline
		\textbf{Gluon} & 8 & \text{starke WW} & \text{Quarks} \\ 
		W^+,\ W^-,\ Z^0 & 3 & \text{schwache WW} & \text{Quarks, Leptonen} \\ 
		\textbf{Photon} & 1 & \text{el.mag. WW} & \text{Quarks, Leptonen}
	\end{array} 
	$
	\caption{Wechselwirkung von Eichbosonen auf Materieteilchen}
\end{table}

\subsection{Erhaltungsgrößen (in nachfolgenden Kapiteln präzisiert)}
Die folgenden Sätze gelten nicht für alle verschiedenen Wechselwirkungen. Welche Wechselwirkungen es gibt, wird später genauer erklärt. Hier soll lediglich eine Sammlung der Erhaltungsgrößen stehen. 

\begin{enumerate}
	\item kinematisch erlaubt, d.h. Energie und Impuls erhalten
	
	\item Leptonenzahl und Baryonenzahl erhalten. Dazu muss man anmerken, dass nur die Neutrino--Oszillation die Baryonenzahlerhaltung verletzt.
	
	\item Gesamtdrehimpuls $(\pvec{J})$ erhalten
	
	\item Parität P und Ladungskonjugation C erhalten (gilt nicht bei schwacher Wechselwirkung)
	
	\item Ladung erhalten
	
	\item Quarkflavour erhalten (gilt nicht bei schwacher Wechselwirkung)
\end{enumerate}

\subsection{Beispiele}
	
	\begin{enumerate}
		\item Positron (Antifermion) + Elektron(Fermion) in Feynman-Diagramm 2. Ordnung:
			\begin{figure}[H]
				\centering
					\begin{tikzpicture}
						\draw [->, >=latex] (0,0) -- (0, 4) node [midway, left] {$t$};
					\end{tikzpicture}	
					%
					\quad
				\begin{subfigure}[b]{0.4\textwidth}
					\feynmandiagram [vertical=a to b] {
						i1 [particle=$e$] -- [fermion] a -- [fermion] i2 [particle=$e$],
						a -- [photon, edge label=$\gamma$] b,
						f1 [particle=$e$] -- [anti fermion] b -- [anti fermion] f2 [particle=$e$],
					};
					%\caption{zeitartig $q^2 > 0$}
				\end{subfigure}
				%
				\begin{tikzpicture}
					\draw [->, >=latex] (0,0) -- (0, 3) node [midway, left] {$t$};
				\end{tikzpicture}	
				%
				\quad
				\begin{subfigure}[b]{0.4\textwidth}
					\feynmandiagram [horizontal=a to b] {
						i1[particle=$e$] -- [fermion] a -- [fermion] i2 [particle=$e$],
						a -- [photon, edge label=$\gamma$] b,
						f1[particle=$e$]-- [anti fermion] b -- [anti fermion] f2 [particle=$e$],
					};
					%\caption{raumartig $q^2 < 0$}
				\end{subfigure}
				\caption{zeitartig $q^2 >0$ (links) / raumartig $q^2 < 0$ (rechts)}
			\end{figure}
			%
			Wichtig: Das Zeitintervall der Wechselwirkung ist durch die Heisenberg'sche Unschärferelation
			\begin{equation}
				 \Delta E \cdot \Delta t \ge \hslashtwo
			\end{equation}
			beschränkt.
			
			Für elektromagnetische WW: $E = h \cdot \nu \Rightarrow \Delta t \ge \frac{1}{4\pi \Delta \nu}$
			
			Diese Unschärfe erlaubt die Superposition aller möglichen Prozesse, die wir durch die Feynman-Diagramme darstellen können.

			\item Compton-Effekt:
				\begin{figure}[H]
					\centering
					\begin{tikzpicture}
						\draw [->, >=latex] (0,0) -- (0, 4) node [midway, left] {$t$};
					\end{tikzpicture}
					%
					\feynmandiagram [vertical=a to b] {
						i1[particle=$e$] -- [anti fermion] a -- [photon] i2 [particle=$\gamma$],
						a -- [anti fermion, edge label=$e$] b,
						f1[particle=$\gamma$]-- [photon] b -- [anti fermion] f2 [particle=$e$],
					};
					\caption{Feynman-Diagramm des Compton-Effekts}
				\end{figure}
			\item Proton-Zerfall:
				\begin{figure}[H]
					\centering
					\begin{tikzpicture}
						\draw [->, >=latex] (0,0) -- (0, 4) node [midway, left] {$t$};
					\end{tikzpicture}
					%
					\begin{tikzpicture}
						\begin{feynman}
							\vertex (p) {$p$};
							\vertex [above=of p] (b);
							\vertex [above right=of b] (c);
							\vertex [above=of b] (n) {$n$};
							\vertex [above right=of b] (d);
							\vertex [above right=of d] (e) {$e^+$};
							\vertex [above=of d] (nue) {$\nu_e$};
							
							\diagram*{
								(p) -- [fermion] (b) -- [fermion] (n);
								(b) -- [scalar, edge label'=$W^+$] (c);
								(e) -- [fermion] (c) -- [fermion] (nue);
								};
						\end{feynman}
					\end{tikzpicture}
					\caption{Feynman-Diagramm des Proton-Zerfalls}
				\end{figure}
	\end{enumerate}
	
	\subsection{Feynman-Kalkül}
	
	Prozesse sind auf beliebig vielen Arten darstellbar (mehrere Vertices möglich). Jeder Vertex liefert zur Übergangswahrscheinlichkeit die Kopplungskonstante $\alpha_{\text{WW}}(q)$, wobei hier $q$ für die Ladung des Teilchens und WW für die Art der Wechselwirkung steht. Wir werden dies im Folgenden \textbf{plausibilisieren}:
	
	Beispielsweise ist es für die elektromagnetische WW die Feinstrukturkonstante $\alpha = \frac{e^2}{\hslash c}$.\\
	
	Je mehr Vertices ein Diagramm enthält, desto geringer ist sein Beitrag zur Gesamtamplitude der superponierten Zustände (alle bis heute erzielten Forschungsergebnisse mit Termen nur bis 4. Ordnung errechnet).
	
	Wir fragen uns nun, wieviel ein Vertex in einem Feynman-Diagramm zur \\ Übergangswahrscheinlichkeit der zugehörigen Reaktion in Abhängigkeit von der Wechselwirkung beiträgt. Dies kann man sich zunächst für die Elektromagnetische Wechselwirkung klar machen. Hier gilt für das Potential:
	\begin{equation}
	V(r) = \frac{e^2}{r} = \frac{\alpha \hslash c}{r}
	\end{equation}
	Nach Fermis Goldener Regel gilt für die Übergangswahrscheinlichkeit P:
	\begin{equation}
	P \propto \frac{dP}{dt} = \lambda = \frac{2\pi}{\hslash} \bra{\psi_E}|V|\ket{\psi_A}|^2 \rho_E \propto  |\bra{\psi_E}|V|\ket{\psi_A}|^2 \propto \alpha^2
	\end{equation}
	Hier wurden bereits 2 Vertices betrachtet, (Anfangszustand und Endzustand). Skaliert man dies auf einen Vertex runter erhält man also für den Beitrag eines Vertex zur Übergangswahrscheinlichkeit: $ P \propto \alpha$ \newline
	Um dieses Prinzip jetzt für andere Teilchen als Elektronen zu verallgemeinern betrachtet man den Beitrag der Ladungszahlen im Coulombpotential. Genauer betrachtet lautet es mit den Ladungszahlen $q_1,q_2$
	\begin{equation}
	V(r) = \frac{q_1 q_2 e^2}{r} = \frac{q_1 q_2 \alpha \hslash c}{r} 
	\end{equation}
	Mit analoger Rechnung halten wir fest, dass pro Vertex die Übergangswahrscheinlichkeit skaliert mit $q_1 q_2 \alpha$ wobei $q_i$ die Ladungszahlen der wechselwirkenden Teilchen sind, (für up Quark z.B $q_1=\frac{2}{3}$ und down Quark $q_2= - \frac{1}{3}$.) \newline
	Wir greifen jetzt ein wenig vorweg:	Betrachtet man die Potentiale der starken und schwachen Wechselwirkung
	\begin{align}
	V_{\text{weak}} &= \frac{g_{\text{weak}}^2}{r} \cdot \exp\lp -\frac{m_{\text{weak}} r}{\hslash c}\rp \\
	V_{\text{strong}} &= \frac{g_{\text{strong}}^2}{r} \cdot \exp\lp -\frac{m_{\text{strong}} r}{\hslash c}\rp
	\end{align}
	So lässt sich auch ein Analogon für diese Wechselwirkungen zur elektromagnetischen Ladung finden. $g_i$ ist analog zu q und somit gilt bei der starken und der schwachen Wechselwirkung für die Wahrscheinlichkeit pro Vertex 
	\begin{equation}
	P \propto \alpha g_1 g_2
	\end{equation}
	Diese Analogie wird bestätigt durch die Wechselwirkungskonstanten für die starke und schwache Wechselwirkung:
	\begin{align}
	\alpha_{weak} &= \frac{q_{weak}^2}{\hslash c} \\
	\alpha_{strong} &= \frac{q_{strong}^2}{\hslash c}
	\end{align}
	Wir halten also letzten Endes fest, dass für eine beliebige Wechselwirkung die Übergangswahrscheinlichkeit pro Vertex gegeben ist durch 
	\begin{equation}
	P \propto x_1 x_2 \alpha
	\end{equation}
	Wobei x die "Ladung" der jeweiligen Wechselwirkung ist. Wir möchten zum Schluss noch betonen, dass dies lediglich eine \textbf{Plausibilisierung} ist und auch die Analogien sehr mit Vorsicht zu genießen sind und stark vereinfacht sind. Außerdem erfolgt eine genauere Betrachtung dieser Wechselwirkungen im weiteren Verlauf des Skripts. \\ \newline
	
\end{document}
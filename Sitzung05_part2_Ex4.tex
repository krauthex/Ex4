\documentclass[Ex4_Zusammenfassung.tex]{subfiles}

\begin{document}
\chapter{Elementarteilchen-Zoo}	
\section{Charakterisierung durch Wechselwirkung}
\textbf{von \hein \& \mitsch} \\

Elementarteilchen lassen sich anhand folgender Größen klassifizieren:
\begin{itemize}
\item Masse
\item Spin
\item Quantenstatistik
\item Ladung
\item Lebensdauer (falls instabil)
\end{itemize}
Teilchen lassen sich  in verschiedene Typen unterteilen: 
\begin{itemize}
	\item Elementarteilchen: 
	\begin{itemize}
		\item unteilbar, keine Struktur oder Anregung
		\item punktförmig $\lp r  <  \SI{1E-18}{m}\rp$ 
		\item z.B: Elektron, Neutrino, Quark
	\end{itemize}
	\item Austausch--/Feldteilchen: vermitteln Wechselwirkung z.B: Photon
	\item Zusammengesetzte Teilchen: 
	\begin{itemize}
		\item gebundene Zustände von Elementarteilchen
		\item z.B: Atom, Proton, Neutron
	\end{itemize}
\end{itemize}

\subsection{Masse}
Anhand des Aston'schen Massenspektrometers lässt sich die Masse von Elementarteilchen experimentell bestimmen. 
Man verwendet dazu eine Kombination von elektrischen und magnetischen Feldern, wobei sich durch die Ablenkung des Teilchens in diesen die Masse bestimmen lässt. 

\begin{figure}[H]
\centering
	\begin{tikzpicture}[scale=1.5]
	\draw [->, >=latex, thin] (0.3, -1.5) -- (0.3, 1.7);
	\draw [->, >=latex, thin] (0.3,0) -- (5,0);7
	\draw [->, >=latex, thin] (1.5,0) -- (1.5, 1.7);
	\draw [->, >=latex, thin] (2.0,0) -- (2.0, 1.7);
	\draw [->, >=latex, thin] (2.5,0) -- (2.5, 1.7);
	
	\node at (0,1.5) (y) {$y$};
	\node at (5.2,0) (x) {$x$};
	
	\node at (3,0.5) {x};
	\node at (3,1) {x};
	\node at (3,1.5) {x};
	\node at (3.5,0.5) {x};
	\node at (3.5,1) {x};
	\node at (3.5,1.5) {x};
	\node at (4,0.5) {x};
	\node at (4,1) {x};
	\node at (4,1.5) {x};
	\node at (4.5,0.5) {x};  
	\node at (4.5,1) {x};
	\node at (4.5,1.5) {x};
	\node at (2,2) {E-Feld};
	\node at (3.5,2) {B-Feld};
	\node at (1.75,-0.8) {$L_E$};
	\node at (3.75,-0.8) {$L_B$};
	\node at (2.2,0.3) {$\theta$};
	
	\draw (2.0,0) arc(0:45:0.5cm);
	\draw (1.5,0) -- (2.8,1.15);
	\draw (4.7,0.7) arc (0:155:1cm);
	\draw (1.5,-0.5) [<->, >=latex, thin]-- (2.5,-0.5);
	\draw (3,-0.5) [<->, >=latex, thin] -- (4.75,-0.5);
		
	\end{tikzpicture}
	\caption{Aufbau eines Massenspektrometers}
\end{figure} 
Nachdem das Teilchen das E--Feld durchlaufen hat ist seine Geschwindigkeit in y--Richtung:
\begin{equation}
	v_y = a_y \cdot t = \frac{F}{m} \cdot \frac{L_E}{v_x}
\end{equation} 
Für den Einfallswinkel des Teilchens folgt: 
\begin{equation}
	\tan \theta = \frac{v_y}{v_x} = \frac{q  E  L_E }{m  v_x}
\end{equation}
Nachdem das Teilchen das B--Feld durchlaufen hat ist seine Geschwindigkeit in y--Richtung:
\begin{equation}
	v_y = v_{0y} + \frac{q B  L_B}{m v_x}
\end{equation}
Durch geeignete Dimensionierung der Apparatur werden Teilchen gleicher Masse unabhängig von Geschwindigkeit gleich stark abgegelenkt. Auf diese Weise lassen sich die Massen stabiler Teilchen bestimmen. 
Bei instabilen Teilchen misst man den Impuls sowie die Flugzeit (mittlere Lebensdauer) und bestimmt die Masse via: 
\begin{align}
	p c &= \beta \gamma m c^2 \\
	t &= \frac{L}{\beta c }
\end{align}
Bei zu kurzen Lebensdauern misst man den Viererimpuls der Zerfallsprodukte,
\begin{equation}
	P^2 = (P_1 + P_2)^2 = \frac{1}{c^2}(E_1 + E_2)^2 - (\vec p_1  + \vec p_2 )^2 = m_x^2 c^2
\end{equation}
wobei man die invariante Masse $ m_x $ im Ruhesystem des Ursprung des Teilchens erhält über: 
\begin{equation}
	m_x = \sqrt{m_1^2 + m_2^2 + 2E_1E_2(1-\beta_{1} \beta_{2} \cos \theta ) }
\end{equation}
Wenn X ein Zerfallsprodukt von Y ist, kann durch die beim Zerfall freiwerdende Energie (Q--Wert) die Masse von X berechnet werden, sofern man die Masse von Y kennt.

\subsection{Spin}
Man kann Teilchen ebenfalls durch ihren Spin klassifizieren. 
Diesen kann man über eine Messung des magnetischen Moments $ \mu $ eines Teilchens bestimmen. Es gilt: 
\begin{align}
	\vec \mu_{s} &= \frac{g_s \mu_0 } {\hslash} \vec S \\
	\mu_{0} &= \frac{e \hslash}{2m_e} \quad \text{Magneton}
\end{align}
Der Faktor g ist in diesem Fall das gyromagnetische Verhältnis , welches individuell vom Teilchen abhängt (für Elektronen: $g_s \approx -2$). Für andere Wechselwirkungen gibt es andere g--Faktoren, doch dazu später.

Man unterscheidet prinzipiell anhand des Spins 2 Teilchenarten: 
\begin{itemize}
\item Teilchen mit halbzahligem Spin: Fermionen 
	\begin{itemize}
		\item Elektronen, Quarks, Neutrinos
		\item Fermi--Dirac--Statistik
	\end{itemize}
\item Teilchen mit ganzzahligem Spin: Bosonen
	\begin{itemize}
		\item Photon, Higgs--Boson
		\item Bose--Einstein--Statistik
	\end{itemize}
Man unterscheidet für die jeweiligen Werte noch verschiedene Arten Bosonen
	\begin{itemize}
		\item $S=0$ \qquad Skalarboson \qquad Pion, Higgs--Boson
		\item $S=1$ \qquad Vektorboson \qquad Photon, Eichboson
		\item $S=2$ \qquad Tensorboson \qquad Graviton (hypothetisch)
	\end{itemize}
\end{itemize} 

\subsection{Klassifizierung anhand Wechselwirkung}
Es gibt 4 fundamentale Wechselwirkungen:
\begin{itemize}
	\item Gravitation
	\item Elektromagnetische Wechselwirkung
	\item Schwache Wechselwirkung (z.B $\beta$-Zerfall)
	\item Starke Wechselwirkung (Bindung von Nukleonen im Kern / Quarks im Nukleon)
\end{itemize}
Die Stärke dieser Wechselwirkungen wird charakterisiert durch eine dimensionlose Kopplungskonstante, die Ladung und die Reichweite.
Wechselwirkungen werden durch Austauschteilchen vermittelt, die allerdings nur aufgrund der Unschärferelation überhaupt existieren können.

Austauschteilchen bleiben in sogenannten virtuellen Zuständen und für die Außenwelt unsichtbar. Jedoch konnten die messbaren physikalischen Prozesse mit diesem Modell mit sonst nicht erreichter Präzision erklärt werden.

\subsubsection{Elektromagnetische Wechselwirkung}
Das Austauschteilchen dieser Wechselwirkung ist das Photon.
Das Potential eines Elektrons, das diese Wechselwirkung auslöst ist gegeben durch:
\begin{equation}
	V(r) = -  \frac{e^2}{r}
\end{equation}
Durch Fouriertransformation vom Orts-- in den Impulsraum kann man dieses Potential auch durch das Minkowsi--Quadrat des Impulsübertragsvektors $q$ darstellen. Das Potential nimmt folgende Form an: 
\begin{equation}
	V(q^2) = \frac{e^2 \hslash^2}{q^2}
\end{equation}
Die Kopplungskonstante für diese Wechselwirkung ist: 
\begin{equation}
	\alpha = \frac{e^2}{\hslash c} = \frac{1}{137}
\end{equation} 

\subsubsection{Schwache Wechselwirkung}
Diese Wechselwirkung wirkt nur auf sehr kleine Abstände, die kleiner als ein Atomradius sind. Sie tritt vorallem bei Zerfällen und Umwandlungen von Teilchen auf (z.B.  $\beta$--Zerfall). Die Austauschteilchen dieser Wechselwirkung sind Eichbosonen ($Z^0,\ W^{+},\ W^{-}$--Boson).
\begin{figure}[H]
	\centering
	\begin{tikzpicture}
		\begin{feynman}
			\vertex (n) {$n$};
			\vertex [above=of n] (b);
			\vertex [above right=of b] (c);
			\vertex [above=of b] (p) {$p^+$};
			\vertex [above right=of b] (d);
			\vertex [above right=of d] (e) {$e^-$};
			\vertex [above=of d] (nue) {$\bar \nu_e$};
							
			\diagram *{
				(n) -- [fermion] (b) -- [fermion] (p),
				(b) -- [scalar, edge label'=$W^-$] (c),
				(c) -- [fermion] (e),
				(nue) -- [fermion] (c)
			};
		\end{feynman}
	\end{tikzpicture}
	\caption{Neutron-Zerfall mit $W^-$ --Boson als Austauschteilchen. Dieser Zerfall ist die Ursache für die $ \beta^- $  --Strahlung.}
\end{figure}
Das Potential dieser Wechselwirkung ist:
\begin{equation}
	V_{\text{weak}} = \frac{g_{\text{weak}}^2}{r} \cdot \exp\lp -\frac{m_{\text{weak}} r}{\hslash c}\rp = \frac{g_{\text{weak}}}{q^2 + m_{\text{weak}}^2}
\end{equation}
Wobei man sich den g--Faktor als Analogon zur Ladung vorstellen kann und mit der Masse $ m_{\text{weak}} $ die Masse der Austauschteilchen gemeint ist, welche groß ist ( $m_{\text{weak}} = \SI{80.4}{GeV}$ für $ W^{\mp} $--Bosonen). 

Aufgrund der Massenbehaftung hat diese Wechselwirkung eine kurze Reichweite 
\begin{align}
	\Delta x &= \SI{2E-18}{m} \quad \text{für}  \quad \beta = 0.02 \\
	\Delta x &= \SI{90E-18}{m} \quad \text{für} \quad \beta = 0.7
\end{align} 
Unabhängig von der Geschwindigkeit $\beta$ der Austauschteilchen kann man also sagen, dass die Größenordnung der Reichweite sehr klein ist. Die Kopplungskonstante ist für diese Wechselwirkung auch klein, was sich darauf zurückführen lässt das die g-Faktoren klein vergleichsweise zur Elementarladung sind. 
\begin{equation}
	\frac{g_{\text{weak}}^2}{\hslash c} = 4 \cdot 10^{-3}
\end{equation} 

\subsubsection{Starke Wechselwirkung}
Diese Wechselwirkung erklärt die Bindung von Quarks in Hadronen. Auch hier wird der Austausch durch Eichbosonen beschrieben, den sog. Gluonen. 
\begin{figure}[H]
	\centering
	\begin{tikzpicture}
		\begin{feynman}
			\vertex (b) ;	
			\vertex [below left=of b] (n1) {$n$};
			\vertex [above left=of b] (p1) {$p$};
			\vertex [right=of b] (c);
			\vertex [above right=of c] (n2) {$n$};
			\vertex [below right=of c] (p2) {$p$};
							
			\diagram*{
				(n1) -- [fermion] (b) -- [fermion] (p1);
				(b) -- [scalar, edge label'=$\pi^-$] (c);
				(p2) -- [fermion] (c) -- [fermion] (n2);
			};
		\end{feynman}
	\end{tikzpicture}
	\caption{Proton--Neutron--Wechselwirkung aufgrund starker Wechselwirkung mit negativ geladenem Pion als Austauschteilchen}
\end{figure}
Das Potential ist ähnlich wie das der schwachen Wechselwirkung:
\begin{equation}
	V_{\text{strong}} = \frac{g_{\text{strong}}^2}{r} \cdot \exp\lp -\frac{m_{\text{strong}} r}{\hslash c}\rp = \frac{g_{\text{strong}}}{q^2 + m_{\text{strong}}^2}
\end{equation}
Die Masse der Austauschteilchen sind  $ m_{\text{strong}} = m_{\pi} = \SI{140}{MeV}$ für Pionen. 

Die Kopplungskonstante ist hier sehr groß:
\begin{equation}
	\frac{g_{\text{strong}}^2}{\hslash c } = 15 
\end{equation}
Die Reichweite ist hier immernoch klein, jedoch 1000--mal größer als bei der schwachen Wechselwirkung. Ab einer Reichweite von $ \sim \SI{2.5}{fm}$ gleichen sich starke Wechselwirkung und Coulomb--Kraft aus. Dies erklärt die Größenordnung von Atomkernen. Die Reichweite beträgt 
\begin{equation}
	\Delta x = \SI{1.4E-15}{m}
\end{equation}

\subsubsection{Wechselwirkung zwischen Quarks}
Allgemeiner lässt sich die starke Wechselwirkung als Wechselwirkung zwischen Quarks auffassen, die Bestandteile der Hadronen sind und Gluonen austauschen. Das Potential hat hier folgende Form: 
\begin{equation}
	V_{\text{strong}} = - \frac{4}{3} \  \frac{g_{\text{strong}}^2}{4 \pi r} + k \cdot r 
\end{equation}
Mit der Konstanten $ k = \SI{1}{GeV/fm} $ welche besagt, wieviel Energie man pro Abstand aufwenden muss um dem Potential entgegenzuwirken. 

\begin{figure}[H]
\centering
	\begin{tikzpicture}[scale=1.5]
	\draw [->, >=latex, thin] (0.3, -1.5) -- (0.3, 1.7);
	\draw [->, >=latex, thin] (0.3,0) -- (3,0);7
	
	\node at (0,1.5) (V) {$V$};
	\node at (3.2,0) (r) {$r$};
	\node at (1,0) {i};
	
	\node at (1.5,-0.125) {$ \sim 0.5 $ fm};
	
	\draw (0.3,-1) to [out=90,in=190] (1,0) -- (2.8,0.8);
	\end{tikzpicture}
	\caption{Potential der Quarkwechselwirkung}
\end{figure} 

Quarks können nicht alleine existieren. Versucht man 2 Quarks zu trennen, so ist das Potential irgendwann so groß, dass aus dieser Energie wieder ein neues Quark-Antiquark-Paar entsteht (Confinement--Hypothese\footnote{\href{https://en.wikipedia.org/wiki/Color\_confinement\#/media/File:Gluon\_tube-color\_confinement\_animation.gif}{https://en.wikipedia.org/wiki/Color\_confinement\#/media/File:Gluon\_tube-color\_confinement\_animation.gif} }).
\end{document}
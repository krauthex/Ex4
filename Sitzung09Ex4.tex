\documentclass[Ex4_Zusammenfassung.tex]{subfiles}


\begin{document}
\section{WdH zu Wirkungsquerschnit, Streuung..}
\textbf{von Michi \& Pauli}\\

Die Wahrscheinlichkeit $w$, dass ein einfallendes Teilchen mit einem Targetteilchen wechselwirkt, errechnet sich aus:
\begin{equation}
	w = \sigma \frac{N_T}{F_T}
\end{equation}
wobei $\sigma$ der Wirkungsquerschnitt,  $F_T$ die bestrahlte Targetfläche und $N_T$ die Anzahl der darin enthaltenen Targetteilchen ist. Also:
\begin{equation*}
	w \propto \text{Target--Teilchenflächendichte}
\end{equation*}
Vorraussetzung hierfür ist:
\begin{equation*}
	\sigma N_T \ll F_T
\end{equation*}

Experimentell im Beschleuniger zu bestimmen ist die Reaktionsrate $W$: 
\begin{equation}
	W = \frac{\md N_{\text{\rom{3}}}}{\md t} = \sigma L
\end{equation}
wobei $L$ die Luminosität des Beschleunigers ist.\\

\subsection*{Differentieller Wirkungsquerschnitt: $\derive{\sigma}{\Omega}$}
\begin{equation}
	\underbrace{j_{\text{Strom}}}_{\mathclap{\nicefrac{1}{\text{(Zeit$\cdot$RWE)} }} } \equiv j_{\text{\rom{3}}} = \derive{\sigma}{\Omega} \lp \Omega \rp \cdot \underbrace{j_{\text{\rom{1}}}}_{\mathclap{\nicefrac{1}{\text{(Zeit$\cdot$Fläche) }} } } 
\end{equation}
wobei RWE für Raumwinkelelement steht. 
\begin{equation}
	\Rightarrow \sigma = \int_{\Omega \setminus \text{Beam} }\derive{\sigma}{\Omega} \md \Omega
\end{equation}
Ebenso gilt allgemein:
\begin{equation}
	I = \mathop{\int \kern -0.7em \int}_{\kern -0.6em F} \vec{j}\ \md \vec{F}
\end{equation}

%%%% Big Fat Drawing incoming!
\begin{figure}[H]
	\centering
	\begin{tikzpicture}
		% Nodes
		\node at (2.8, 0.4) (beam) {\tiny Beam};
		\node at (4.0, 2.5) (omega3) {$\Omega_{\text{\rom{3}}}$};
		\node at (0, 1.4) (duenn) {dünn};
		\node at (-1.7, 0.9) (ft) {$F_T$};
		\draw (ft) -- (-1.2, 0.6);
		
		% Shapes
		\draw (0,0) circle (4cm);
		\draw (-0.75,0) ellipse (0.5cm and 1.2cm);
		\draw (-0.75,1.2) -- (0.75, 1.2);
		\draw (-0.75,-1.2) -- (0.75, -1.2);
		\draw [rotate=180] (-0.75,-1.2) arc (270:90:0.5cm and 1.2cm) ;
		\draw [rotate around={-47:(2.8,2.64575)}, pattern=north west lines] (2.8,2.64575) ellipse (1.2cm and 0.5cm);
		
		%Arrows
		\draw [->, >=latex] (-4.5,0) node [above] {$p_0$} node [below] {$E_0$} -- (-0.75,0) node [midway, above] {$j_{\text{\rom{1}}},\ L=N_{\text{\rom{1}}} \cdot j_{\text{\rom{1}}}$};
		\draw [->, >=latex] (1.25,0) -- (4.5,0) node [xshift=0.9cm] {geht durch} node [anchor=south, yshift=-0.6cm] {$j_{\text{\rom{2}}}$};
		\draw [->, >=latex] (0,0) -- (3,2.64575) node [midway, above, sloped] {$j_{\text{\rom{3}}} $} node [midway, below, sloped] (gstrt) {\tiny (gestreut)};
		\draw (2,0) to [bend right] (gstrt.south) node [yshift=-0.5cm]{$\theta$};
		
		%Beam
			\foreach \y in {-0.2, 0.2}
				\draw [dotted] (-3.99, \y) -- (3.1, \y);
		\draw [dotted, rotate around={180:(3,-0.95)}, yshift=0.37cm, xshift=-0.4cm] (3, -0.2) arc (120:270:0.5cm and 1cm);
		\draw [dotted, rotate around={180:(3,0.2)}, yshift=0cm, xshift=-0.16cm] (3, 0.2) arc (90:240:0.5cm and 1cm);
	\end{tikzpicture}
	\caption{Streuung eines Teilchenstrahles (Beam) an einem Target bzw. an einer Targetfläche ($F_T$). }
\end{figure}

\subsection*{Rutherford-Querschnitt}
\begin{equation}
	\derive{\sigma^R}{\Omega} = \lp \frac{zZe^2}{4 E_0 \sin^2 \lp \frac{\vartheta}{2} \rp} \rp^2 
\end{equation}

\subsection*{Zusammenhang zw. Streu--/Raumwinkel und Impulsübertrag $|q|$}
\begin{align}
	\vec{q} &= \vec{p} - \pvec{p}^\prime \\
	|\vec{q}| &= 2 |\vec{p}| \sin \lp \frac{\vartheta}{2} \rp 
\end{align}
\textbf{Herleitung:} Impulserhaltung wegen
\begin{enumerate}
	\item Target sehr schwer
	\item Reibung vernachlässigbar (elastischer Stoß)
\end{enumerate}
\begin{equation}
	\vec{p} = \threevec{0}{0}{|\vec{p}|}, \quad \pvec{p}^\prime = \threevec{|\vec{p}| \sin \vartheta \cos \varphi}{|\vec{p}| \sin \vartheta \sin \varphi}{|\vec{p}| \cos \vartheta}
\end{equation}
\begin{align}
	\left\Vert \vec{q} \right\Vert^2 = \lp \vec{p} - \pvec{p}^\prime \rp^2 &= \left\vert \vec{p} \right\vert^2
	% left parenthesis, not scaling to underbrace:
	\lp \vphantom{\sin^2 \vartheta \cos^2 \varphi + \sin^2 \vartheta \sin^2 \varphi } \right.
	\sin^2 \vartheta \cos^2 \varphi + \sin^2 \vartheta \sin^2 \varphi + \underbrace{1 - 2\cos \vartheta + \cos^2 \vartheta}_{\text{Binom.}} 
	% right parenthesis, not scaling to underbrace
	\left. \vphantom{\sin^2 \vartheta \cos^2 \varphi + \sin^2 \vartheta \sin^2 \varphi} \rp\\
	\intertext{Vernachlässigung der quadratischen Terme führt auf:}
	%% next equation
	&= 2 \left\vert \vec{p} \right\vert^2 \lp 1 - \cos \vartheta \rp = 4 \left\vert \vec{p} \right\vert^2 \sin^2 \lp \frac{\vartheta}{2} \rp
\end{align}
Hierbei wurde verwendet, dass
\begin{equation*}
	1 - \cos \vartheta = 2 \sin^2 \lp \frac{\vartheta}{2} \rp
\end{equation*}

\subsection*{Raumwinkel $\Omega$}
\begin{equation}
	\Omega := \mathop{\int}_{\varphi_1}^{\varphi_2} \mathop{\int}_{\vartheta_1}^{\vartheta_2} \sin \vartheta^\prime \md \vartheta^\prime \md \varphi^\prime
\end{equation}
für Rutherford folgt hieraus: 
\begin{align}
	\Rightarrow \md \Omega &= \underbrace{\frac{\partial \Omega}{\partial \varphi} \md \varphi}_{=0} + \frac{\partial \Omega}{\partial \vartheta} \md \vartheta\\
	&= 2 \pi \sin \vartheta \md \vartheta
\end{align}
\begin{equation}
	\int_{S^2} \md \Omega = 4\pi \approx 12.57\si{sr}
\end{equation}
wobei $\si{sr}$ für Steradianten steht.

\subsection*{Differentieller Wirkungsquerschnitt Part \rom{2}}
\begin{equation}
	\derive{\sigma}{q^2} = \derive{\sigma}{\Omega} \derive{\Omega}{q^2} = \derive{\sigma}{\Omega} \frac{2 \pi \sin \vartheta \md \vartheta}{\md\lp 4 p^2 \sin^2 \frac{\vartheta}{2} \rp} = \derive{\sigma}{\Omega} \frac{\pi}{\left\vert \vec{p} \right\vert^2}
\end{equation}
\begin{equation}
	\text{Herleitung:}
	\lp \derive{\vartheta}{('')} = \lp \derive{\lp 4p^2 \sin^2 \lp \frac{\vartheta}{2} \rp \rp}{\vartheta} \rp^{-1} = 4p^2 \sin \frac{\vartheta}{2} \cos \frac{\vartheta}{2} = 2 p^2 \sin \vartheta \rp 
\end{equation}
	


\end{document}



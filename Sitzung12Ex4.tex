\documentclass[Ex4_Zusammenfassung.tex]{subfiles}

\begin{document}
\chapter{Symmetrien und Erhaltungsgrößen}
\textbf{von \mitsch \& \soeren} \\ 

Als Symmetrie bezeichnet man eine Invarianz von Zuständen oder physikalischen Gesetzen(Bewegungsgleichungen) unter Symmetrietransformationen.
Beispiele hierfür sind:
\begin{itemize}
\item Bewegungsgl. im Zentralfeld invariant unter Rotation
\item Relativistische Bewegungsgl. invariant unter Lorentztransformation
\item Schrödingergl. invariant unter Phasentransformation
\end{itemize}
In diesem Zusammenhang wurde von Emmy Noether das folgende Theorem entwickelt:
\begin{tcolorbox}[title= Noether-Theorem,colback=red!50!white]
Zu jeder Symmetrietransformation existiert eine Erhaltungsgröße. \newline
Zu jeder Erhaltungsgröße existiert eine Symmetrietransformation
\end{tcolorbox}

Wir betrachten die zeitliche Veränderung des Erwartungswerts einer Observarblen (repräsentiert durch den Operator O) in einem System mit dem Hamiltonian H. Falls der Operator O zeitunabhängig ist gilt:
\begin{equation}
\partial_t \braket{\hat{O}} = \frac{1}{i\hslash} \braket{[O,H]}
\end{equation}
\textbf{Beweis:}
	\begin{align*}
	\partial_t \braket{\hat{O}} &= \ \partial_t \braket{\psi(t)|\hat{O}|\psi(t)} \\ 
	&=\braket{\partial_t \psi(t)|\hat{O}|\psi(t)} + \braket{\psi(t)|\hat{O}|\partial_t \psi(t)} \\ \stackrel{^{SGL}_{\bra{\psi} \  = \ (\ket{\psi})^{\dag}}}{=} &-\frac{1}{i\hslash} \braket{\psi(t)|H\hat{O}|\psi(t)} + \frac{1}{i\hslash} \braket{\psi(t)|\hat{O} H |\psi(t)} \\ 
	&= \frac{1}{i\hslash} \braket{\psi(t)|[O,H]|\psi(t)}
	\end{align*}
Das bedeutet, dass eine Observarble O eine Erhaltungsgröße ist, falls $[H,O] = 0$.  In diesem Fall gibt es erhaltene Quantenzahlen. 
\section{Symmetrietransformation}
Eine Symmetrietransformation U ist einen umkehrbare, simultane (wirkt auf alle Vektoren des Raums), eindeutige Abbildung der Zustände und Operatoren, die alle physikalische Aussagen des Systems unverändert lässt.
\begin{align}
\ket{\psi} &\longmapsto \ket{\psi'} = U \ket{\psi} \\
\hat{O} &\longmapsto \hat{O}' = U \hat{O} U^{\dag}
\end{align}

\begin{itemize}
\item Falls U unitär ist d.h. $U U^{\dag} = \mathds{1}$, ist die Norm des Zustands $\ket{\psi}$ erhalten.
\item Eigenvektoren des Operators O mit reellem Eigenwert o sind auch Eigenvektoren des Operators O' mit Eigenwert o.
\begin{equation}
\hat{O}' \ket{\psi'} = U \hat{O}'  U^{\dag} U \ket{\psi} = U \hat{O} \ket{\psi} = U o \ket{\psi} = o \ket{\psi'}
\end{equation}
\item $ [U,H] = 0 \longleftrightarrow U \ket{\psi}$ erfüllt SGL. 
\end{itemize}
Wir suchen also im folgenden unitäre Transformationen U die mit dem Hamiltonian H kommutieren. Hierfür betrachten wir im folgenden zwei Fälle:
\subsection{Diskrete Transformation, U hermitesch}
Wählen wir die Observarble selbst als Transformation $U=O$ so ist Hermitizität gewährleistet, d.h. $\braket{O} = \braket{O}^{*}$ \newline Da die Definitionsbereiche von O und $O^{\dag}$ übereinstimmen folgt hieraus dass $O=O^{\dag}$ \newline Kombinieren wir dies mit der Unitaritätsbedingung so erhalten wir 
\begin{equation}
O^2 = OO^{\dag} = \mathds{1}
\end{equation}
Da wir $[U,H]=[O,H]=0$ fordern, existiert eine gemeinsame Eigenvektorbasis von O und H deren Elemente $\ket{\psi}$ wir im folgenden betrachten.
Wir können an dieser Stelle 2 wichtige Folgerungen ziehen:
\begin{enumerate}
\item $O^2 \ket{\psi} = \ket{\psi} = o^2 \ket{\psi} \longrightarrow o = \pm 1$
\item $ \braket{\psi|\hat{O}|\psi} = \braket{\psi' \hat{O}^{\dag}|\hat{O}|\hat{O}^{\dag} \psi'} =  \braket{\psi'|\hat{O} \hat{O} \hat{O}^{\dag}|\psi'} = \braket{\psi'|\hat{O}|\psi'}$
\end{enumerate}
Bei $U=O$ handelt es sich um eine \textbf{diskrete Transformation}, das heißt, dass nur eine diskrete Verschiebung der Variable generiert werden kann. Beispiele für diskrete Transformationen sind 
\begin{itemize}
\item Translation um eine Gitterkonstante bei einem kubischen Gitterkristall
\item Paritätstransformation  (mehr dazu später)
\item Raumspiegelung,Zeitspiegelung
\item Ladungskonjugation
\end{itemize}
\textbf{Erinnerung Produktzustände} \newline
Betrachtet man Zustände aus verschiedenen Hilberträumen $\ket{\psi}_i \in \mathcal{H}_i$ so kann man diese über das Tensorprodukt koppeln zu einem Produktzustand:
\begin{equation}
(\psi_1,\psi_2,...,\psi_N) \mapsto \prod_{i=1}^N \ket{\psi}_i \in \bigotimes_{i=1}^N \mathcal{H}_i
\end{equation}
Wir fragen uns nun wie Operatoren die auf dem Hilbertraum $\mathcal{H}_j$ definiert sind auf Zustände aus diesem Tensorprodukt wirken. Man definiert die Operation auf dem Produktraum wie folgt
\begin{equation}
O \prod_{i=1}^N \ket{\psi}_i = O \ket{\psi}_j \prod_{j\neq i} \ket{\psi}_i
\end{equation}

Betrachten wir einen Produktzustand $\ket{\psi} = \ket{\psi}_1 \ket{\psi}_2$ (z.B 2 Teilchen), wobei $\ket{\psi}_{1,2}$ Eigenzustände zur Observarblen O sind,und transformieren diesen Zustand mit unserer simultanen Symmetrietransformation U=O, so wirkt diese Transformation sowohl auf $\ket{\psi}_1$ als auch $\ket{\psi}_2$ da sie simultan ist.
\begin{equation}
\ket{\psi'} = U\ket{\psi} = O\ket{\psi}_1 \ O\ket{\psi}_2 = o_1 \ket{\psi}_1 \ o_2 \ket{\psi}_2 = o_{ges} \ket{\psi}
\end{equation}
Dies lässt sich für einen Produktzustand aus N Hilberträumen $\ket{\psi} = \prod_{i=1}^N \ket{\psi}_i$ induktiv fortführen. Identifiziert man jeden Eigenwert $o_i$ eines Eigenzustands $\ket{\psi}_i \in \mathcal{H}_i$ mit einer Quantenzahl $n_i$ so erhalt man eine \textbf{multiplikative Erhaltungsgröße n } als Produkt der Quantenzahlen $n = \prod_{i=1}^N n_i$.

\subsection{Kontinuierliche Transformation,U nicht hermitesch}
Falls U nicht hermitesch ist, kann es einen hermitschen Operator (Generator) $\hat{O}$ geben, sodass
\begin{equation}
	U = \mathds{1} + i \epsilon \hat{O}
\end{equation}
O kann hier auch wieder eine Observarble sein, die per Definition die Hermitizitätsbedingung erfüllt. Diese Transformation U erfüllt außerdem unsere Unitaritätsbedingung wie man wie folgt sieht:
\begin{equation}
	U^{\dag} U = (\mathds{1} - i \epsilon \hat{O}^{\dag}) (\mathds{1} + i \epsilon \hat{O}) = 1 + i \epsilon \underbrace{(\hat{O} - \hat{O}^{\dag})}_{=0} + \mathcal O(\epsilon^2) = \mathds{1} 
\end{equation}
Da auch hier $[U,H]=0$ gefordert wird folgt $[O,H]=0$. Es existiert also auch hier eine gemeinsame Basis von Eigenvektoren von H und O und da auch $[U,O]=0$ gilt folgt:
\begin{equation}
	\hat{O}' = U \hat{O} U^{\dag} = U U^{\dag} \hat{O} = \hat{O}
\end{equation}
Wir schließen daraus, dass die Erwartungswerte von O unter dieser Symmetrietransformation erhalten bleiben.\\ \newline
Man kann eine finite Transformation um $\Delta$ als n-fache Serie von infinitesimal kleinen Transformation um $\epsilon$ darstellen mit
\begin{equation}
	\Delta = \lim_{n \rightarrow \infty} \lim_{\epsilon \rightarrow 0} n \cdot \epsilon
\end{equation}
Wir erhalten als finite Symmetrietransformation U um $\Delta$
\begin{equation}
	U(\Delta)= \lim_{n\rightarrow \infty} \left( \vphantom{\frac{\Delta}{n}} \right. %
	\mathds{1} + i \underbrace{\frac{\Delta}{n}}_{\epsilon}%
	\left. \vphantom{\frac{\Delta}{n}} \right)^n = e^{i\Delta \hat{O}}
\end{equation}
Analog zur diskreten Transformation betrachten wir wieder einen Produktzustand aus 2 Eigenzuständen von O (z.B 2 Teilchen) $\ket{\psi} = \ket{\psi}_1 \ket{\psi}_2 \in \mathcal{H}_1 \bigotimes \mathcal{H}_2$ und lassen unsere Symmetrietransformation auf diesen Zustand wirken.
\begin{equation}
	U \ket{\psi} = U \ket{\psi}_1 U \ket{\psi}_2 = e^{i\Delta(\hat{O}_1 + \hat{O}_2)} \ket{\psi}_1 \ket{\psi}_2
\end{equation}
Wobei $\hat{O}_i = \hat{O}_{|_{\mathcal{H}_i}}$ \newline
Identifizieren wir wieder zu jedem Eigenzustand $\ket{\psi}_i \in \mathcal{H}_i$ mit Eigenwert $o_i$ bezüglich der Observarblen $\hat{O}_i$ eine Quantenzahl $n_i$ so bekommen wir eine \textbf{additive Erhaltungsgröße n} als Summe der Quantenzahlen $ n = \sum_i n_i$ \\ \newline
Beispiele für kontinuierliche Symmetrietransformationen sind:
\begin{itemize}
\item Infinitesimale Raumtranslationen, Infinitesimale Rotationen
\item Zeitentwicklungen, die die Schrödinger-Gleichung unverändert lassen
\end{itemize}

\section{Beispiele}
\subsection{Ladungserhaltung}
In der Elektrodynamik ist die Ladungserhaltung verbunden mit einer Eichinvarianz der elektromagnetischen Potentiale $\phi$ und A unter einer Eichtransformation über das Differential eines Skalarfelds $\chi(\vec x,t)$
\begin{align}
\phi & \mapsto \phi' = \phi - \frac{\partial \chi}{\partial t} \\
A & \mapsto A' = A + \nabla \chi 
\end{align}
In der Quantenmechanik entspricht diese Eichinvarianz einer Phasenverschiebung des Zustands $\ket{\psi}$ um einen Phasenfaktor $e^{i\chi}$
\begin{equation}
\psi \mapsto \psi' = e^{i\chi} \psi
\end{equation}
Diese Symmetrietransformation ist möglich, dass es sich bei der Elektrodynamik um eine renormierbare Feldtheorie handelt. Die Potentiale sind langreichweitig, sodass die zur Theorie gekoppelte Größe - die Ladung - erhalten ist.

\subsection{Baryonenzahl}
\begin{itemize}
\item (Anti)-Quarks: Baryonenzahl $A = \pm \frac{1}{3}$
\item (Anti)-Baryonen: $A= \pm 1$
\item Mesonen: $A=0$ 
\end{itemize}
Nach der Großen vereinheitlichten Theorie (GUT) ist die Baryonenzahl keine exakte Erhaltungsgröße, so dass Protonen mit der Zeit zerfallen, allerdings mit einer sehr großen Halbwertszeit ($\tau(p \rightarrow e^+ + \pi^0) > 10^{33} a$).
Falls die Baryonenzahl als Folge einer lokalen Eichinvarianz erhalten sein sollte, erwarte wir damit verbunden ein koppelndes langreichweitiges Feld. Es gibt keine experimentelle Evidenz für ein solches Feld.
Ein Baryonenkoppelfeld würde das Verhältnis von gravitativer Masse zu träger Masse verändern,dafür gibt es jedoch keinen Hinweis $\frac{\Delta R}{R} < 10^{-12}$.

\subsection{Leptonenzahl}
Für Leptonen($e^-,\mu-,\tau^-$) ist die Leptonenzahl $L=\pm 1$. Diese Zahl ist für jede Generation konstant, wenn man Neutrinooszillationen herausnimmt. Hierbei taucht die Frage auf ob Neutrino und Antineutrino verschieden sind, welcher wir bald auf den Grund gehen werden.
\end{document}
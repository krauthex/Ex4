\documentclass[Ex4_Zusammenfassung.tex]{subfiles}

\begin{document}
\section{Paritätsverletzung in der schwachen Wechselwirkung}
\subsubsection*{von \mitsch \& \anton}


\subsection{Helizität}
Die Helizität ist definiert als die Projektion des Spins auf die Bewegungsrichtung (Impulsrichtung). Sie ist gegeben durch
\begin{equation}
	h:=\frac{\vec{s}\cdot \vec{p}}{\abs{\vec{p}}\cdot \abs{\vec{s}}}
\end{equation}
Sie hat folgende Eigenschaften:
\begin{itemize}
	\item Die Helizität ist eine quantenmechanische Observable
	\item Sie ist eine Erhaltungsgröße (im selben Bezugssystem)
	\item Sie besitzt die Parität -1 und die C--Parität 1 (darauf wird später eingegangen)
\end{itemize}

\subsection{Chiralität}

	
	
\end{document}